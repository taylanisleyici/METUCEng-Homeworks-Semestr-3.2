\documentclass[10pt,a4paper, margin=1in]{article}
\usepackage{fullpage}
\usepackage{amsfonts, amsmath, pifont}
\usepackage{amsthm}
\usepackage{graphicx}
\usepackage{float}

\usepackage{tkz-euclide}
\usepackage{tikz}
\usepackage{pgfplots}
\pgfplotsset{compat=1.13}

\usepackage{geometry}
 \geometry{
 a4paper,
 total={210mm,297mm},
 left=10mm,
 right=10mm,
 top=10mm,
 bottom=10mm,
 }
 % Write both of your names here. Fill exxxxxxx with your ceng mail address.
 \author{
  İşleyici, Osman Taylan\\
  \texttt{e2449496@ceng.metu.edu.tr}
  \and
  LastName2, FirstName2\\
  \texttt{exxxxxxx@ceng.metu.edu.tr}
}

\title{CENG 384 - Signals and Systems for Computer Engineers \\
Spring 2023 \\
Homework 3}
\begin{document}
\maketitle



\noindent\rule{19cm}{1.2pt}

\begin{enumerate}

\item %write the solution of q1

\item %write the solution of q2  
	\begin{enumerate}
    % Write your solutions in the following items.
    \item $\sum\limits_{\forall l}a_l*a_{k-l}$ because of the multiplication property of fourier series.
    \item Fourier series coefficient of even part of $x(t)$ is equal to real part of $a_k = \mathbb{R} \{a_k\}$.
	\item $x(t-t_0) = a_ke^{-jk(w\pi/T)t_0}$, $x(t+t_0) = a_ke^{jk(2\pi/T)t_0}$. $x(t-t_0) + x(t+t_0) = a_ke^{-jk(w\pi/T)t_0} + a_ke^{jk(2\pi/T)t_0}$.
    \end{enumerate}

\item %write the solution of q3

\item %write the solution of q4
    \begin{enumerate}   
    % Write your solutions in the following items.
    \item $sin(w_0t) = \frac{j}{2}(-e^{iw_0t} + e^{-iw_0t})$, $2*cos(w_0t) = e^{iw_0t} + e^{-iw_0t}$, $cos(2w_0t + \pi/4) = (e^{j\pi/4}*e^{2iw_0t} + e^{-2iw0t}/e^{j\pi/4})/2$ \\
    $a_{-2} = \frac{1}{2\sqrt{j}}$, since $\sqrt{j} = (1+j)/\sqrt{2}$, $a{-2} = \frac{1+j}{2\sqrt{2}}$\\
    $a_{-1} = j/2+1$\\
    $a_0 = 1$\\
    $a_1 = 1-j/2$\\
    $a_2 = \frac{1}{2\sqrt{j}} = \frac{\sqrt{2}}{2+2j}$\\

    \begin{tikzpicture}
        \begin{axis}[    xlabel={$k$},    ylabel={$|a_k|$},    axis lines=middle,    xmin=-2.5, xmax=2.5,    ymin=-0.1, ymax=1.5,    xtick={-2,-1,0,1,2},    ytick={0,0.5,1,1.118,1.5},    yticklabels={0,0.5,1,$\frac{\sqrt{5}}{2}$,1.5},    enlargelimits=true,    clip=false]
        \addplot [ycomb, black, thick, mark=*] table [x={k}, y={|a_k|}] {data4ai.dat};
        \end{axis}
        \end{tikzpicture}

        \begin{tikzpicture}
            \begin{axis}[    xlabel={$k$},    ylabel={$\angle a_k$},    axis lines=middle,
                    xmin=-2.5, xmax=2.5,    ymin=-1.5, ymax=1.5,    xtick={-2,-1,0,1,2},    
                    ytick={-0.7853, -0.5235, 0.5235, 0.7853},
                    yticklabels={$-\pi/4$, $-\pi/6$, $\pi/6$, $\pi/4$},]
            \addplot [ycomb, black, thick, mark=*] table [x={k}, y={angle}] {data4aii.dat};
            \end{axis}
            \end{tikzpicture}
    
    
    \item $\dot{y}(t) + y(t) = x(t)$, we should first find the particular solution of this system. We should write $x(t)$ as $e^{\lambda t}u(t)$ and $y(t)$ as $Kx(t)$ than solve the equation for K. $(\lambda K  + K)e^{\lambda t}u(t) = e^{\lambda t} u(t)$, $\lambda K + K = 1$, $K = \frac{1}{1+\lambda}$. The pole of transfer function is the eigenvalue of system. Which is -1 for this question.
	\item %write the solution of q4c
    \item %write the solution of q4d
    \end{enumerate}

\item %write the solution of q5
    \begin{enumerate}
    % Write your solutions in the following items.
    \item %write the solution of q5a
    \item %write the solution of q5b
	\item %write the solution of q5c
	\item %write the solution of q5d
    \end{enumerate}    
    
\item %write the solution of q6
    \begin{enumerate}
    % Write your solutions in the following items.
    \item We can see that the period of this signal is 4. We can then we can use the analysis formula to find fourier coefficients.\\ 
    $a_k = \frac{1}{4}\sum\limits_{n=0}^{3}x[n]e^{-jk(\pi/2)n}$\\
    $a_k = \frac{1}{4}(0+e^{-jk\pi/2} + 2e^{-jk\pi} + e^{-3jk\pi/2})$ \\
    $e^{jk\pi/2} = j$ so\\
    $a_0 = 1$\\
    $a_1 = -1/2$\\
    $a_2 = 0$\\
    $a_3 = -1/2$
    From periodicity of signal, we can say that $a_k$ does also have period 4.\\
    \begin{filecontents}{data6a.dat}
        k    a_k
        -4   1
        -3   0.5
        -2   0
        -1   0.5
        0    1
        1    0.5
        2    0
        3    0.5
        4    1
    \end{filecontents}\\
    \begin{tikzpicture}
        \begin{axis}[
          xlabel={$n$},
          ylabel={$x[n]$},
          axis lines=middle,
          ymax=1.5,
          xmin=-4, xmax=4,
          ymin=-0.3, ymax=1.5,
          ytick={-1, -0.5, 0, 0.5, 1},
          ]
          \addplot [ycomb, black, thick, mark=*] table [x={k}, y={a_k}] {data6a.dat};
        \end{axis}
        \end{tikzpicture}
    \item If we examine the graph, we can see that this signal is $y[n] = x[n]x[n-1]$.\\
    We can use difference and multiplication properties to find the spectral coefficients of Fourier series. \\
    $x[n+1] \leftrightarrow c_k = a_ke^{jk\pi/2} = \frac{1}{4}(e^{jk\pi}+2e^{jk\pi/2}+1)$\\
    $x[n]x[n+1] \leftrightarrow b_k=\sum\limits_{l=0}^4 a_lc_{k-l}$\\
    Or we can just use the analysis formula instead of calculating this sum.\\
    $b_k = \frac{1}{4}\sum\limits_{n=0}^4y[n]e^{-jk(\pi/2)(n-2)}$\\
    $b_k = \frac{1}{4}(e^{jk\pi/2}+2)$\\
    $e^{j\pi/2} = j$\\
    $b_0 = 3/4$\\
    $b_1 = \frac{j+2}{4}$\\
    $b_2 = 1/4$\\
    $b_3= \frac{4-j}{2}$.

    \end{enumerate}
    
\item %write the solution of q7
    \begin{enumerate}
    % Write your solutions in the following items.
    \item %write the solution of q7a
    \item %write the solution of q7b
    \end{enumerate}    
	
\item %write the solution of q8	

\end{enumerate}


\end{document}

