\chapter{Architectural Views}

\section{Context View}

\subsection{Stakeholders' uses of this view}

There are three main stakeholders of the system: the volunteers, victims, developers. \\

The volunteers use this view to understand how they will help the victims by sharing information with website.
The victims use this view to understand how they can get help to get food, healthcare, etc.
The developers use this view to understand how they can build a bridge between victims and volunteers and how they can achieve to forward the validated information as fast as possible.

\subsection{Context Diagram}

\begin{figure}[H]
    \includegraphics[scale = 0.6]{assets/Context Diagram.png}
    \caption[Context Diagram for afetbilgi.com]{Context Diagram for afetbilgi.com}
\end{figure}

As it can be observed from the context diagram, the system interacts with 3 external entities. These are users, admins and information validators. Afetbilgi.com users can use this website for 2 purpose. It can be used to deliver help or it can be used to get help who are affected by the disaster.
in addition to that, information validator gets the all information which delivered to them and by reaching the authorities validates the information. In meanwhile, admins update the database with these validated datas in the system.

\subsection{External Interfaces}

In this section, the external interfaces of the afetbilgi.com will be provided.

\begin{figure}[H]
    \includegraphics[scale = 0.4]{assets/externalInterfaces.png}
    \caption[External Interfaces Class Diagram for Afetbilgi]{External Interfaces Class Diagram for Afetbilgi}
\end{figure}

\begin{center}
    \begin{table}[H]
        \begin{tabular}{| m{6cm}| m{8cm} |}
            \hline
            \textbf{Operation} & \textbf{Description} \\
            \hline
            generatePdf & The website generate the pdf which contains the selected city crucial information in compact form as a downloadable file by users.\\
            \hline
            showLocationsOnMaps & Website shows the important locations on the map.\\
            \hline
            changeLanguage & Users can choose a language between Turkish, English, Kurdi and Arabic.\\
            \hline
            selectCity & Website filter the datas according to selected city.\\
            \hline
            showInformations & Shows the information as subparts on the website mainpage.\\
            \hline
            editCode & Developers can edit the code in the source files.\\
            \hline
            updateDatabase & Developers can update the database with the validated information.\\
            \hline
            getData & Users can reach the validated data by using this website.\\
            \hline
            sendEmail & In case of truth of information or error in server, users can send an email.\\
            \hline
            pushRequest & Developers can push the code changes to Github Repository.\\
            \hline
            pullRequest & Developers can pull the code from the Github Repository.\\
            \hline
            clone & Developers can download the code from the Github Repository.\\
            \hline
            commit & Developers can record changes in their code.\\
            \hline
            log & Developers can look at it previous commits.\\
            \hline
            getClickableIcons & Google Maps API gets the icons such as hospital and pharmacies to show on map.\\
            \hline
        \end{tabular}
        \caption[External Interface Operation Descriptions]{External Interface Operation Descriptions}
    \end{table}
\end{center}

\subsection{Interaction scenarios}

\begin{figure}[H]
    \includegraphics[scale = 0.5]{assets/Activity DiagramMaps.png}
    \caption[Activity Diagram for Google Maps API Afetbilgi Interactions]{Activity Diagram for Google Maps API Afetbilgi Interactions}
\end{figure}

\begin{figure}[H]
    \includegraphics[scale = 0.5]{assets/Activity DiagramEditingCode.png}
    \caption[Activity Diagram for Developer and Github Editing Code Interactions]{Activity Diagram for Developer and Github Editing Code Interactions}
\end{figure}

\section{Functional View}

\subsection{Stakeholders' use of this view}

There are three main stakeholders of the system: the volunteers, victims and developers. \\

The volunteers use this view to find locations which can help to go there or donating money.
The victims especially use this view to make use of Google Maps API since it is very helpful to find all crucial locations in a compact form near them.
The developers use this view to add new functionalities to this system so that they can reach more people and help to faster delivery of aid.

\subsection{Component Diagram}

\begin{figure}[H]
    \includegraphics[scale = 0.4]{assets/ComponentDiagram.png}
    \caption[Component Diagram for Afetbilgi]{Component Diagram for Afetbilgi}
\end{figure}
~\\~\\
Afetbilgi.com has three main subsystem which are Amazon Web Services, Database and Computer.

\begin{itemize}
    \item Web Service consists of three parts, Amazon Web Services, CloudWatch and LogBucket.
    \item Computer has a part which is browser. It allows to search in internet and gets the information with a graphical userface.
    \item When the user reach the webpage via browser, comptuer sends a request to the web services. Webservice holds their database in a two different file type. It requests these files and parser them to send the information to the users. 
    \item Database uses two different file type which are JSON file and YAML file.
    \item GitHub is an external component which stores the source code of the system. Moreover, when the developers make a change in their system, it allows them to update the whole webservice by Github. 
    \item When the users try to do tasks which uses Google Maps API or PDF API, webservice interacts with these APIs to created necessary information for the user.
    \item The interface between the computer and webservice uses a https protocol to communication. HTTPS is encrypted in order to increase security of data transfer.
\end{itemize}

\subsection{Internal Interfaces}

In this section, the internal interfaces of afetbilgi will be provided. \\

\begin{figure}[H]
    \includegraphics[scale = 0.4]{assets/InternalInterfaces.png}
    \caption[Internal Interface of Afetbilgi]{Internal Interface of Afetbilgi}
\end{figure}

\begin{center}
    \begin{table}[H]
        \begin{tabular}{| m{6cm}| m{8cm} |}
            \hline
            \textbf{Operation} & \textbf{Description} \\
            \hline
            sendData & Sends a data to a web server.\\
            \hline
            updateData & Update the current database with new data.\\
            \hline
            getRequest & Retrieves a request maded by a server.\\
            \hline
            requestData & Requests data from a server.\\
            \hline
            getData & Retrieves data from a server.\\
            \hline
            htmlParser & Parses and extracts information from HTML documents.\\
            \hline
            cssParser & Parses and extracts information from CSS documents.\\
            \hline
            JavascriptEngine & Executes Javascript code.\\
            \hline
            EventHandling & Handles events triggered by user interactions or system events.\\
            \hline
            ControlConnection & Manages and maintains connections for controlling or communicating with external systems.\\
            \hline
            sendNotification & Sends a notification via email to the user.\\
            \hline
            storeLogData & Stores log data for future analysis.\\
            \hline
            storeMetricsData & Stores metrics or performance data for analysis.\\
            \hline
            visualizeMetrics & Shows metrics or performance data in a visual format.\\
            \hline
            sendAlarms & Sends alarms to notify about specific conditions.\\
            \hline
            storeData & Saves data for future use.\\
            \hline
            runServer & Runs a server.\\
            \hline
            storeLogs & Saves log entries or records for tracking or analysis.\\
            \hline
            encryption & Implements encryption algorithms or techniques to secure data.\\
            \hline
            createLog & Creates log entries or records for system.\\
            \hline
            controlLogs & Manages log entries, including filtering or searching.\\
            \hline
            processInformation & Helps to analyzes information.\\
            \hline
        \end{tabular}
        \caption[Internal Interface Operation Descriptions]{Internal Interface Operation Descriptions}
    \end{table}
\end{center}

As shown in Figure 4.6, the internal interfaces of Afetbilgi are Amazon Web Service, CloudWatch, LogBucket, Database and Browser. The operations of interfaces can be described in below. \\

\begin{itemize}
    \item There are two different API which is implemented into to webservice. According to users' needs the necessary API used to create the PDF file which contains information or shows the important locations on Google Maps.
    \item There are two internal interface which has an association with the Amazon Web Service which are CloudWatch and LogBucket. With these internal interfaces, it stores the log of website and give an opportunity of analyzing the metrics of website.
\end{itemize}

\subsection{Interaction Patterns}

\begin{figure}[H]
    \includegraphics[scale = 0.7]{assets/SequenceDiagramUpdateDB.png}
    \caption[Sequence Diagram of Update Database]{Sequence Diagram of Update Database}
\end{figure}

\begin{figure}[H]
    \includegraphics[scale = 0.6]{assets/SequenceDiagramOpenBrowser.png}
    \caption[Sequence Diagram of Get Data]{Sequence Diagram of Get Data}
\end{figure}

\begin{figure}[H]
    \includegraphics[scale = 0.5]{assets/SequenceDiagramMaps.png}
    \caption[Sequence Diagram of Opening Map]{Sequence Diagram of Opening Map}
\end{figure}

\section{Information View}
This view focuses on describing the information requirements Afetbilgi.com. Data operations are examined in this view.

\subsection{Stakeholders' uses of this view}
\begin{itemize}
    \item The volunteers use this view to understand which information they should send while submitting something to website.
    \item The victims use this view to understand what is the meaning of the information they get from website.
    \item The developers use this view to successfully add, edit and delete data from website.
\end{itemize}

\subsection{Database Class Diagram}
\begin{figure}[H]
    \includegraphics[scale = 0.4]{assets/DatabaseClassDiagram1.png}
    \caption[Database Class Diagram Of Afetbilgi.com]{Database Class Diagram Of Afetbilgi.com}
\end{figure}

\subsection{Operations on Data}
\begin{table}[H]
    \begin{tabular}{|p{6cm}|p{10cm}|}
        \hline
        \textbf{Operation}   & \textbf{Description}                                                                                                                                    \\
        \hline
        \hline
        CreateGeneralNeeding & Crete a new general needing with given attributes returns result.                                                                                                       \\
        \hline
        DeleteGeneralNeeding                & Deletes general needing with given id returns true if successfully deleted.                                                                                                        \\
        \hline
        GetGeneralNeedings                  & Returns a list of general needings in given cityID. \\
        \hline
        CreateCategory                & Creates a category with given name returns success state.                                                                                   \\
        \hline
        DeleteCategory          & Deletes a category and its subcategories returns success state.\\
        \hline
        CreateSubCategory & Creates a subcategory in given category with given name returns result of operation.\\
        \hline
        DeleteSubCategory & Deletes a subcategory with its needings and services returns true if successfully deleted, otherwise returns false.\\
        \hline
        FilterSubCategory & Returns a list of general needings or healthcare services in a city with given subcategory, for example hospitals in Hatay.\\
        \hline
        AddCity & Adds a city with given id and name.\\
        \hline
        RemoveCity & Removes a city with healthcare services and general needings associated with the city.\\
        \hline
        AddHealthService & Adds a new healthcare service with given name, city, district, location and subcategory.\\
        \hline
        RemoveHealthService & Removes the healthcare service with given id.\\
        \hline
        GetHealthServices & Returns healthcare services in a city.\\
        \hline
    \end{tabular}
    \caption[Operations on Data]{Operations on Data}
\end{table}


\section{Deployment View}
This view focuses on how Afetbilgi.com is physically deployed in terms of hardware and software.

\subsection{Stakeholders' uses of this view}
\begin{itemize}
    \item Volunteers can use this view to understand how the system is deployed.
    \item Developers can use this view while updating and upgrading Afetbilgi.com.
\end{itemize}

\subsection{Deployment Diagram}
\begin{figure}[H]
    \includegraphics[scale = 0.5]{assets/DeploymentDiagram1.png}
    \caption[Deployment Diagram]{Deployment Diagram}
\end{figure}

\section{Design Rationale}

\begin{itemize}
    \item Application serves as a bridge between victims and volunteers.
    \item Systems setted up considering communication helps to improve viablity of information and speed of communication.
    %Functional view da iletişim halinde kurulan sistemlerle verinin doğruluğu ve ulaşım hızı sağlanıyor
    \item Database should be kept with MySQL and operations of database should be handled with MySQL node.js bridge.
    \item A statical version of website and pdfs should be kept and deployed in cloudflare servers for speeding up website.
\end{itemize}